% This is samplepaper.tex, a sample chapter demonstrating the
% LLNCS macro package for Springer Computer Science proceedings;
% Version 2.20 of 2017/10/04
%
\documentclass[runningheads]{llncs}
%
\usepackage[top=5cm, bottom=5.6cm, left=4.5cm, right=4.2cm]{geometry}
\usepackage{graphicx}
\usepackage{array}
\newcolumntype{P}[1]{>{\centering\arraybackslash}p{#1}}
% Used for displaying a sample figure. If possible, figure files should
% be included in EPS format.
%
% If you use the hyperref package, please uncomment the following line
% to display URLs in blue roman font according to Springer's eBook style:
% \renewcommand\UrlFont{\color{blue}\rmfamily}

\makeatletter
\renewcommand\paragraph{\@startsection{paragraph}{4}{\z@}%
                                    {3.25ex \@plus1ex \@minus.2ex}%
                                    {-1em}%
                                    {\normalfont\normalsize\bfseries}}
\makeatother
% my packages and commands
\graphicspath{{figs/}}
%% The amsthm package provides extended theorem environments
%\usepackage{amsthm}
%\usepackage[numbers,sort&compress]{natbib}
\usepackage{booktabs} % for nice tables
\usepackage{xcolor}
\usepackage{subcaption} % for subfigures
\usepackage[colorlinks=true,citecolor=blue]{hyperref}
\newcommand{\ud}{\mathrm{d}}
\renewcommand{\vec}[1]{\mathbf{#1}}
\newcommand{\veca}[2]{\mathbf{#1}{#2}}
\newcommand{\bm}[1]{\mathbf{#1}}
\newcommand{\etal}{et al.}
\begin{document}
%
\title{Vectorization of the code for guided wave propagation problems}
%
\titlerunning{Vectorization of the code for GW propagation problems}
% If the paper title is too long for the running head, you can set
% an abbreviated paper title here
%
\author{Pawel Kudela\inst{}\orcidID{0000-0002-5130-6443}  \and 
Piotr Fiborek\inst{}\orcidID{[0000-0002-5030-3312} 
}
%
\authorrunning{P. Kudela and P. Fiborek}
% First names are abbreviated in the running head.
% If there are more than two authors, 'et al.' is used.
%
\institute{Institute of Fluid-Flow Machinery, Polish Academy of Sciences, 80-231 Gdansk, Poland
\email{pk@imp.gda.pl}}

%
\maketitle              % typeset the header of the contribution
%
%\begin{abstract}
\paragraph{Abstract.}
Vectorization of the code for simulation of guided wave propagation problems based on the spectral element method is presented. 
In the code, flat shell spectral elements are utilized for spatial domain representation.
The implementation is realised by using Matlab Parallel Computing Toolbox and optimized for Graphics Processing Unit (GPU) computation. 
In this way, considerable computation speed-up can be achieved in comparison to computation on conventional processors. 
The implementation includes an interpolation of wave-field on a uniform grid. 
The method was tested on experimental full wave-field data measured by scanning laser Doppler vibrometer. 
Good agreement between numerical and experimental results was achieved. 
Due to relatively short computation time, large data sets can be generated by using the proposed implementation. 
The large data sets are especially useful for deep neural network training or other soft computing methods opening up new possibilities in health monitoring of metallic and composite structures.

\keywords{Spectral Element Method \and Guided Waves \and Code Vectorization \and GPU computation.}
%\end{abstract}
%
\\[2em]
%
\section{Introduction}
The motivation of this work was the need for the development of large data set to be used for Machine Learning purposes. 
It consists of 475 examples of various delamination sizes and locations in a composite plate of dimensions 500 \(\times\) 500 mm.
For each example simulation of guided wave propagation and interaction with delamination for selected excitation signal is needed.
 
Guided wave propagation problems are computationally demanding.
Usually, a very dense mesh is necessary to model short wavelengths (at least five nodes per wavelength are required to represent the shape of wave).
To date, there is no commercial software available which could be used for efficient wave propagation simulation. 
This fact is confirmed by studies conducted by Leckey \etal~\cite{Leckey2018}. 
They investigated four numerical simulation tools: custom implementation of the 3D Elastodynamic Finite Integration Technique (EFIT) \cite{Schubert1998} along with three widely used commercial finite element codes: COMSOL, ABAQUS, and ANSYS. 
The investigated example was related to the interaction of propagating Lamb waves with delaminations in cross-ply laminates. 
The laminate was modelled by a fine mesh of 3D solid elements. 
The numerical results were compared with experiments in terms of the wave-field showing quite good agreement in case of COMSOL, ABAQUS Implicit and ANSYS implicit. 
Unfortunately, despite the simulations were performed on a workstation equipped with 16 cores, the efficiency of each investigated methods is so low that it is prohibitive to perform any parametric study or generate large data sets (the shortest simulation run time was for the case of COMSOL i.e. 19.5~hours followed by ABAQUS implicit i.e. 40~hours).

The guided wave propagation modelling problem has been tackled by using various methods over the past few decades. 
The following methods can be included: analytic methods~\cite{Giurgiutiu2014}, semi--analytic methods~\cite{Bartoli2006,Gravenkamp2014},  analytical and higher order finite element hybrid approach for 2D analysis~\cite{Vivar-Perez2014}, the frequency domain spectral finite element method~\cite{Doyle1989,RoyMahapatra2003},  the wavelet spectral finite element~\cite{Mitra2008,Yang2016}, the time domain spectral element method~\cite{Lonkar2013,Ostachowicz2012,Schulte2010}, the spectral cell method~\cite{Duczek2014} and  the Local Interaction Simulation Approach~\cite{Kijanka2013}.
Some of these methods have been recently implemented for the use on Graphics Processing Units (GPU) in order to decrease computation time ~\cite{Kijanka2013,Kudela2016,Mossaiby2019,Shen2017}.
The advantage of such approach is staggering computation speedup in comparison to the use of CPU.

The method presented in this paper for solving guided wave propagation problems combines the high order time domain spectral element method (SEM) with the Compute Unified Device Architecture (CUDA),  through Matlab Parallel Computing Toolbox. 
The presented concept of parallel implementation of SEM is similar to the parallel implementation developed previously~\cite{Kudela2016} but it is applied to flat shell spectral elements instead of 3D solid elements. 
Therefore, the computation can be performed faster than in case of utilisation of 3D solid spectral elements. 

\section{Code vectorization concept}
In classic finite element approach elemental matrices are assembled to form equation of motion:
\begin{equation}
\bm{M} \vec{\ddot{U}} + \bm{C} \vec{\dot{U}} + \bm{K} \vec{U} = \vec{F}, \label{eq:motion}
\end{equation}  
where \( \bm{M} \) is the global mass (inertia) matrix, \( \bm{K} \) is the global stiffness matrix,  \(\bm{C} \) is the global damping matrix, \(\vec{U}\) is the vector of global degrees of freedom and~\(\vec{F}\) is the vector of the time-dependent excitation (in this particular case the vector of equivalent piezoelectric forces). 
The most efficient way to solve the (\ref{eq:motion}) is by using explicit integration scheme. 
Assuming the central difference method:
\begin{equation}
\ddot{\vec{U}}\simeq \frac{1}{\Delta t^2} \left(\vec{u}_{t+\Delta t} - 2\,\vec{u}_t + \vec{u}_{t-\Delta t}\right), \label{eq:central_scheme}
\end{equation}
\begin{equation}
\dot{\vec{U}}\simeq \frac{\vec{u}_{t+\Delta t} -\vec{u}_{t-\Delta t}}{2 \Delta t}
\label{eq:first_derivative_scheme}
\end{equation}
and substituting (\ref{eq:central_scheme})-(\ref{eq:first_derivative_scheme}) into (\ref{eq:motion}) leads to:
\begin{equation}
	\underbrace{\left(\frac{1}{\Delta t^2} \,\bm{M} + \frac{1}{2 \Delta t} \bm{C}\right)}_{\vec{M}_0} \vec{u}_{t+\Delta t} = \vec{F}_t - \underbrace{\left(\bm{K} \vec{u}_t\right)}_{\vec{F}^i} + \underbrace{\left(\frac{2}{\Delta t^2} \,\bm{M} \right)}_{\vec{M}_1}\vec{u}_t 
	+ \underbrace{\left(- \frac{1}{\Delta t^2} \,\bm{M} + \frac{1}{2 \Delta t} \bm{C}\right)}_{\vec{M}_2} \vec{u}_{t-\Delta t}.
\label{eq:explicit_integration}
\end{equation}

It should be underlined that due to the orthogonality of shape functions and application of the Gauss-Lobatto-Legendre (GLL) integration rule the mass matrix is diagonal. 
It has been shown that for the Lamb wave attenuation modelling it is possible to assume that the damping matrix is proportional to mass matrix~\cite{Wandowski2017}. In such case calculation of displacements at the time step \(t + \Delta t\) is straightforward and does not require costly matrix inversion. 
However, the term \(\vec{F}^i=\bm{K}\vec{u}_t\) related to internal forces at the time step \(t\) is still computationally intensive. 
Moreover, assembly of the stiffness matrix is troublesome because it requires a lot of memory and limits the size of wave propagation problems which can be simulated. 
In order to alleviate these deficiencies, a parallel code is proposed in which assembly is performed at the internal force vector level without the necessity of stiffness matrix assembly. 
The proposed approach is very similar to the parallel implementation given in~\cite{Kudela2016}. 

The proposed approach differs in the calculation of elemental forces which depend on the contribution of the extensional stiffness, the flexural stiffness, bending-stretching coupling,  twisting-stretching along with bending-shearing coupling, stre\-tching-shearing coupling and bending-twisting coupling instead of the matrix of elastic constants assigned to each layer of a composite laminate. 
Hence, the proposed method is more suitable for wave propagation modelling in multilayer composite laminates because it leads to a much lower number of degrees of freedom. 

Essentially, the term \(\bm{K}\vec{u}_t\) is expanded for each element by using sparse matrices of shape function derivatives \(\bm{N}_{\xi}^e,\,\bm{N}_{\eta}^e\), components of the inverse of Jacobian matrix \((\vec{J}^{-1})_{ij}^e\), elastic constants \(\vec{Q}_{ij}^e\) integrated over the thickness, integration weights \(\vec{W}^e\) and nodal displacement vector at the element level \(\hat{\vec{u}}_0^{e} \). 
These matrices and vectors are combined together in the form corresponding to disjoint spectral elements as:
\begin{equation}
\bm{N},_{\xi} = \left[
\begin{array}{cccc}  
\bm{N},_{\xi}^{e=1} & 0 & \ldots & 0\\[2pt]
0& \bm{N},_{\xi}^{e=2}  & \ldots& 0\\[2pt]
\vdots&\vdots&\ddots&0\\[2pt]
0& 0 &0&\bm{N},_{\xi}^{e=n}\\[2pt]
\end{array}\right],\quad
\vec{U}_x = \left[
\begin{array}{c}  
\hat{\vec{u}}_0^{e=1}  \\[2pt]
\hat{\vec{u}}_0^{e=2} \\[2pt]
\vdots\\[2pt]
\hat{\vec{u}}_0^{e=n}\\[2pt]
\end{array}\right].
\end{equation}
Therefore, at the final step, it is necessary to assemble the global force vector. 
It can be performed according to mesh colouring algorithm proposed in~\cite{Kudela2016}. 
The algorithm uniformly divides the nodes of spectral elements within the whole mesh into 12 sets. 
The sets are of the same size so the computation can be perfectly balanced between workers or resources can be uniformly divided within one graphics card.
Due to the fact, that the colouring algorithm requires that the number of nodes must be divided into 12 sets of equal size it is best to have 36-node spectral elements in the mesh.

Once internal forces \(\vec{F}^i\) are calculated and substituted into (\ref{eq:explicit_integration}), the displacements at time step \(t+\Delta t\) can be explicitly obtained from a perfectly vectorized code:
\begin{equation}
\vec{u}_{t+\Delta t}=1./\vec{M}_0\, .*\left(\vec{F}_t - \vec{F}^i +\vec{M}_1 \, .* \vec{u}_t +\vec{M}_2 \, .* \vec{u}_{t-\Delta t}\right),
\label{eq:vectorized_motion}
\end{equation} 
in which terms \(\vec{M}_0\), \(\vec{M}_1\) and \(\vec{M}_2\) are stored as vectors and \(./\) is element-wise division  and \(.*\) denotes element-wise operation known as Hadamard product (the same symbol for element-wise operation is used in Matlab). 
In particular, all components in (\ref{eq:vectorized_motion}) are implemented in Matlab Parallel Computing Toolbox as \verb|gpuArray|. 
In this way, the implementation is simple whereas CUDA GPU computation is transparent to the user. 

Depending on the size of the problem,  calculations are about 5--12 times faster on GPU than on a single CPU.

\section{Results}
The numerical results were validated by experimental wave-field data acquired by scanning laser Doppler vibrometer.
The investigated specimen was made out of unidirectional CFRP laminate with an orientation angle of reinforcing fibres 90\(^{\circ}\).
Material properties of single-layer CFRP laminate are given in Tab.~\ref{tab:mat_prop}.
The mass density was 1574.1 kg/m\textsuperscript{3}.
The dimensions of the specimen were 1200\(\times\)1200 mm and the thickness was 2.85 mm.
A piezoelectric transducer of diameter 10 mm was placed at the centre of the plate. 
Three excitation frequencies were considered 16.5 kHz, 50 kHz and 100 kHz.
The signal had a form of sinusoid modulated by Hann window (5 cycles).
The measurements were taken on a lower left quarter of the composite laminate on the opposite side with respect to the piezoelectric transducer.
Measurements were acquired at a regular grid of 491\(\times\)491 points.

Numerical simulations were carried out with the same parameters as in the experiment.
The wave-field data at a quarter of the plate was interpolated on a regular grid of points of the same size as in the experimental data.

It should be added that damping was not included in the numerical simulations.
\begin{table}[h!]
		\renewcommand{\arraystretch}{1.3}
	\caption{Material properties of the investigated unidirectional CFRP laminate; Units: GPa.}
	\begin{center}
			\begin{tabular}{cccccc} 
			%\hline
			\toprule
			$Q_{11}$ & $Q_{12}$  & $Q_{22}$ & $Q_{44}$ & $Q_{55}$ & $Q_{66}$\\
			% \cmidrule(lr){1-3} \cmidrule(lr){4-6} \cmidrule(lr){7-7}
			%\hline
			\midrule
			120& 5.6& 12.7 & 3.1 & 5.3 & 4.5\\
			%\hline 
			\bottomrule 
		\end{tabular} 
	\end{center}
		\label{tab:mat_prop}
\end{table}

Comparative results are presented in Figs.~\ref{fig:wavefield16_5}--\ref{fig:wavefield100}.
It can be seen that qualitative agreement between numerical and experimental wave-fields is very good especially for A0 mode which has the greatest amplitude.

In experimental wave-fields at 50 kHz additional low-amplitude waves can be observed which correspond to the faster S0 mode (see Fig.~\ref{fig:wavefield50a}).
Unfortunately, S0 mode is not visible in numerical simulations. 
However, this problem can be alleviated by tuning in-plane and out-of-plane damping matrix components.

When 100 kHz excitation signals are applied, more guided wave modes can be observed in experimental signals (Figs.~\ref{fig:wavefield100b},~\ref{fig:wavefield100d},~\ref{fig:wavefield100f}).
The model is not able to properly simulate higher guided wave modes because it is based on the first-order shear deformation theory. 
It has not enough degrees of freedom per node in order to properly model through-thickness guided wave behaviour at higher frequencies.
\begin{figure} [h!]
	\centering
	\begin{subfigure}[b]{0.49\textwidth}
		\centering
		\includegraphics[]{figure1a.png}
		\caption{Numerical}
		\label{fig:wavefield16_5a}
	\end{subfigure}
	\begin{subfigure}[b]{0.49\textwidth}
		\centering
		\includegraphics[]{figure1b.png}
		\caption{Experimental}
		\label{fig:wavefield16_5b}
	\end{subfigure}
	\begin{subfigure}[b]{0.49\textwidth}
		\centering
		\includegraphics[]{figure1c.png}
		\caption{Numerical}
		\label{fig:wavefield16_5c}
	\end{subfigure}
	\begin{subfigure}[b]{0.49\textwidth}
		\centering
		\includegraphics{figure1d.png}
		\caption{Experimental}
		\label{fig:wavefield16_5d}
	\end{subfigure}
	\begin{subfigure}[b]{0.49\textwidth}
		\centering
		\includegraphics{figure1e.png}
		\caption{Numerical}
		\label{fig:wavefield16_5e}
	\end{subfigure}
	\begin{subfigure}[b]{0.49\textwidth}
		\centering
		\includegraphics{figure1f.png}
		\caption{Experimental}
		\label{fig:wavefield16_5f}
	\end{subfigure}
	\caption{Wave-field of propagating guided waves for the excitation frequency \textbf{16.5 kHz} at the time instances:  0.25 (a)-(b), 0.5 (c)-(d) and 0.75 (e)-(f) ms. }
	\label{fig:wavefield16_5}
\end{figure}
\begin{figure} [h!]
	\centering
	\begin{subfigure}[b]{0.49\textwidth}
		\centering
		\includegraphics[]{figure2a.png}
		\caption{Numerical}
		\label{fig:wavefield50a}
	\end{subfigure}
	\begin{subfigure}[b]{0.49\textwidth}
		\centering
		\includegraphics[]{figure2b.png}
		\caption{Experimental}
		\label{fig:wavefield50b}
	\end{subfigure}
	\begin{subfigure}[b]{0.49\textwidth}
		\centering
		\includegraphics[]{figure2c.png}
		\caption{Numerical}
		\label{fig:wavefield50c}
	\end{subfigure}
	\begin{subfigure}[b]{0.49\textwidth}
		\centering
		\includegraphics{figure2d.png}
		\caption{Experimental}
		\label{fig:wavefield50d}
	\end{subfigure}
	\begin{subfigure}[b]{0.49\textwidth}
		\centering
		\includegraphics{figure2e.png}
		\caption{Numerical}
		\label{fig:wavefield50e}
	\end{subfigure}
	\begin{subfigure}[b]{0.49\textwidth}
		\centering
		\includegraphics{figure2f.png}
		\caption{Experimental}
		\label{fig:wavefield50f}
	\end{subfigure}
	\caption{Wave-field of propagating guided waves for the excitation frequency \textbf{50 kHz} at the time instances:  0.25 (a)-(b), 0.5 (c)-(d) and 0.75 (e)-(f) ms. }
	\label{fig:wavefield50}
\end{figure}
\begin{figure} [h!]
	\centering
	\begin{subfigure}[b]{0.49\textwidth}
		\centering
		\includegraphics[]{figure3a.png}
		\caption{Numerical}
		\label{fig:wavefield100a}
	\end{subfigure}
	\begin{subfigure}[b]{0.49\textwidth}
		\centering
		\includegraphics[]{figure3b.png}
		\caption{Experimental}
		\label{fig:wavefield100b}
	\end{subfigure}
	\begin{subfigure}[b]{0.49\textwidth}
		\centering
		\includegraphics[]{figure3c.png}
		\caption{Numerical}
		\label{fig:wavefield100c}
	\end{subfigure}
	\begin{subfigure}[b]{0.49\textwidth}
		\centering
		\includegraphics{figure3d.png}
		\caption{Experimental}
		\label{fig:wavefield100d}
	\end{subfigure}
	\begin{subfigure}[b]{0.49\textwidth}
		\centering
		\includegraphics{figure3e.png}
		\caption{Numerical}
		\label{fig:wavefield100e}
	\end{subfigure}
	\begin{subfigure}[b]{0.49\textwidth}
		\centering
		\includegraphics{figure3f.png}
		\caption{Experimental}
		\label{fig:wavefield100f}
	\end{subfigure}
	\caption{Wave-field of propagating guided waves for the excitation frequency \textbf{100 kHz} at the time instances:  0.2 (a)-(b), 0.3 (c)-(d) and 0.4 (e)-(f) ms. }
	\label{fig:wavefield100}
\end{figure}
\clearpage
\section{Conclusions}
A novel vectorized code for guided wave propagation problems was developed.
It is based on the time domain spectral element method in which flat shell elements are utilized.
The proposed code is implemented for the use on GPU which results in 5-12 times computation speed-up in comparison to computations on CPU.

Qualitative results in terms of full wave-filed data are satisfactory.
The model is limited to the modelling of fundamental guided wave modes.
Therefore, discrepancies between numerical and experimental results at higher frequencies are expected. 

Further studies are needed in relation to the optimisation of damping parameters and quantitative estimation of differences between numerical and experimental signals.

\section*{Acknowledgements}
The research was funded by the Polish National Science Center under grant agreement no 2018/31/B/ST8/00454. 
P. Kudela would like to acknowledge the Polish National Agency for Academic Exchange for the support in the frame of the Bekker Programme (PPN/BEK/2018/1/00014/DEC/1). 
Authors are also grateful to Task-CI for allowing the use of Matlab and Parallel Computing Toolbox licences. 
%
% ---- Bibliography ----
%
% BibTeX users should specify bibliography style 'splncs04'.
% References will then be sorted and formatted in the correct style.
%
 \bibliographystyle{splncs04}
 \bibliography{EWSHM2020-code-vectorization}
%

\end{document}
