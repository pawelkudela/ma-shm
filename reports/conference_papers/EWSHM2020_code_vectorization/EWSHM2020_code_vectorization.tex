% This is samplepaper.tex, a sample chapter demonstrating the
% LLNCS macro package for Springer Computer Science proceedings;
% Version 2.20 of 2017/10/04
%
\documentclass[runningheads]{llncs}
%
\usepackage[top=5cm, bottom=5.6cm, left=4.5cm, right=4.2cm]{geometry}
\usepackage{graphicx}
\usepackage{array}
\newcolumntype{P}[1]{>{\centering\arraybackslash}p{#1}}
% Used for displaying a sample figure. If possible, figure files should
% be included in EPS format.
%
% If you use the hyperref package, please uncomment the following line
% to display URLs in blue roman font according to Springer's eBook style:
% \renewcommand\UrlFont{\color{blue}\rmfamily}

\makeatletter
\renewcommand\paragraph{\@startsection{paragraph}{4}{\z@}%
                                    {3.25ex \@plus1ex \@minus.2ex}%
                                    {-1em}%
                                    {\normalfont\normalsize\bfseries}}
\makeatother
% my packages and commands
\graphicspath{{figs/}}
%% The amsthm package provides extended theorem environments
% \usepackage{amsthm,bm}
\usepackage[numbers,sort&compress]{natbib}
\usepackage{booktabs} % for nice tables
\usepackage{xcolor}
\usepackage[colorlinks=true,citecolor=blue]{hyperref}
\newcommand{\ud}{\mathrm{d}}
\renewcommand{\vec}[1]{\mathbf{#1}}
\newcommand{\veca}[2]{\mathbf{#1}{#2}}
\newcommand{\bm}[1]{\mathbf{#1}}
\newcommand{\etal}{et al.}
\begin{document}
%
\title{Vectorization of the code for guided wave propagation problems}
%
\titlerunning{Vectorization of the code for GW propagation problems}
% If the paper title is too long for the running head, you can set
% an abbreviated paper title here
%
\author{Pawel Kudela\inst{}\orcidID{0000-0002-5130-6443}  \and 
Piotr Fiborek\inst{}\orcidID{[0000-0002-5030-3312} 
}
%
\authorrunning{P. Kudela and P. Fiborek}
% First names are abbreviated in the running head.
% If there are more than two authors, 'et al.' is used.
%
\institute{Institute of Fluid-Flow Machinery, Polish Academy of Sciences, 80-231 Gdansk, Poland
\email{pk@imp.gda.pl}}

%
\maketitle              % typeset the header of the contribution
%
%\begin{abstract}
\paragraph{Abstract.}
Vectorization of the code for simulation of guided wave propagation problems based on the spectral element method is presented. 
In the code, flat shell spectral elements are utilized for spatial domain representation.
The implementation is realised by using Matlab Parallel Computing Toolbox and optimized for Graphics Processing Unit (GPU) computation. 
In this way, considerable computation speedup can be achieved in comparison to computation on conventional processors. 
The implementation includes an interpolation of wavefield on a uniform grid. 
The method was tested on experimental full wavefield data measured by scanning laser Doppler vibrometer. 
Good agreement between numerical and experimental results was achieved. 
Due to relatively short computation time, large data sets can be generated by using the proposed implementation. 
The large data sets are especially useful for deep neural network training or other soft computing methods opening up new possibilities in health monitoring of metallic and composite structures.

\keywords{Spectral Element Method \and Guided Waves \and Code Vectorization \and GPU computation.}
%\end{abstract}
%
\\[2em]
%
\section{Introduction}
The motivation of this work was the need for the development of large data set to be used for Machine Learning purposes. 
It consists of 475 examples of various delamination sizes and locations in a composite plate of dimensions 500 \(\times\) 500 mm.
For each example simulation of guided wave propagation and interaction with delamination for selected excitation signal is needed.
 
Guided wave propagation problems are computationally demanding.
Usually, a very dense mesh is necessary to model short wavelengths (at least five nodes per wavelength are required to represent the shape of wave).
To date, there is no commercial software available which could be used for efficient wave propagation simulation. 
This fact is confirmed by studies conducted by Leckey \etal~\cite{Leckey2018}. 
They investigated four numerical simulation tools: custom implementation of the 3D Elastodynamic Finite Integration Technique (EFIT) \cite{Schubert1998} along with three widely used commercial finite element codes: COMSOL, ABAQUS, and ANSYS. 
The investigated example was related to the interaction of propagating Lamb waves with delaminations in cross-ply laminates. 
The laminate was modelled by a fine mesh of 3D solid elements. 
The numerical results were compared with experiments in terms of the wavefield showing quite good agreement in case of COMSOL, ABAQUS Implicit and ANSYS implicit. 
Unfortunately, despite the simulations were performed on a workstation equipped with 16 cores, the efficiency of each investigated methods is so low that it is prohibitive to perform any parametric study or generate large data sets (the shortest simulation run time was for the case of COMSOL i.e. 19.5~hours followed by ABAQUS implicit i.e. 40~hours).

The guided wave propagation modelling problem has been tackled by using various methods over the past few decades. 
The following methods can be included: analytic methods~\cite{Giurgiutiu2014}, semi--analytic methods~\cite{Bartoli2006,Gravenkamp2014},  analytical and higher order finite element hybrid approach for 2D analysis~\cite{Vivar-Perez2014}, the frequency domain spectral finite element method~\cite{Doyle1989,RoyMahapatra2003},  the wavelet spectral finite element~\cite{Mitra2008,Yang2016}, the time domain spectral element method~\cite{Schulte2010,Ostachowicz2012,Lonkar2013}, the spectral cell method~\cite{Duczek2014} and  the Local Interaction Simulation Approach~\cite{Kijanka2013}.
Some of these methods have been recently implemented for the use on Graphics Processing Units (GPU) in order to decrease computation time ~\cite{Kijanka2013,Kudela2016,Shen2017,Mossaiby2019}.
The advantage of such approach is staggering computation speedup in comparison to the use of CPU.

The method presented in this paper for solving guided wave propagation problems combines the high order time domain spectral element method (SEM) with the Compute Unified Device Architecture (CUDA),  through Matlab Parallel Computing Toolbox. 
The presented concept of parallel implementation of SEM is similar to the parallel implementation developed previously~\cite{Kudela2016} but it is applied to flat shell spectral elements instead of 3D solid elements. 
The proposed approach differs in the calculation of elemental forces which depend on the contribution of the extensional stiffness, the flexural stiffness, bending-stretching coupling,  twisting-stretching along with bending-shearing coupling, stretching-shearing coupling and bending-twisting coupling instead of the matrix of elastic constants assigned to each layer of a composite laminate. 
Hence, the proposed method is more suitable for wave propagation modelling in multilayer composite laminates because it leads to a much lower number of degrees of freedom. 
Therefore, the computation can be performed faster than in case of utilisation of 3D solid spectral elements. 

\section{Code vectorization concept}

\section{Results}
\begin{table}
		\renewcommand{\arraystretch}{1.3}
	\caption{Material properties of the investigated unidirectional CFRP laminate; Units: GPa.}
	\begin{center}
			\begin{tabular}{cccccc} 
			%\hline
			\toprule
			$C_{11}$ & $C_{12}$  & $C_{22}$ & $C_{44}$ & $C_{55}$ & $C_{66}$\\
			% \cmidrule(lr){1-3} \cmidrule(lr){4-6} \cmidrule(lr){7-7}
			%\hline
			\midrule
			120& 5.6& 12.7 & 3.1 & 5.3 & 4.5\\
			%\hline 
			\bottomrule 
		\end{tabular} 
	\end{center}
		\label{tab:mat_prop}

\end{table}

\section{Conclusions}
\section*{Acknowledgements}
The research was funded by the Polish National Science Center under grant agreement no 2018/31/B/ST8/00454. 
P. Kudela would like to acknowledge the Polish National Agency for Academic Exchange for the support in the frame of the Bekker Programme (PPN/BEK/2018/1/00014/DEC/1). 
Authors are also grateful to Task-CI for allowing the use of Matlab and Parallel Computing Toolbox licences. 
%
% ---- Bibliography ----
%
% BibTeX users should specify bibliography style 'splncs04'.
% References will then be sorted and formatted in the correct style.
%
 \bibliographystyle{splncs04}
 \bibliography{EWSHM2020-code-vectorization}
%

\end{document}
